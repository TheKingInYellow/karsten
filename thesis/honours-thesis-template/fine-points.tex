
%---------------------------------------------------------
% Chapter: A chapter about typesetting.  See https://xkcd.com/1015/
%----------------------------------------------------------

\chapter{Some Fine Points of Typesetting}

No-one expects an honours or masters thesis to be a world-class
example of typography (unless, I suppose, you are studying that
discipline).
However, that doesn't mean you shouldn't put a tiny bit more effort in
to make it look as nice as possible.  \TeX\ already takes care of many
of the details for you, but you can help \TeX\ out every now and then.

\section{Non-Breakable Spaces}

In some situations you may want to tell \TeX\ to \emph{not} put a
line-break between two particular words.  For example, you would not
want ``Section'' to be at the end of one line and ``3.2.1'' to be at
the beginning of the next line.  Similarly you don't want ``Figure'',
``Table'', ``Algorithm'', ``Listing'', and so on at the end of one
line and the corresponding number at the beginning of the next line.
(Especially if the next line is at the top of the next page!)

Any time you want to keep to words together, use a ``\verb|~|''
instead of a space between the two words.  For example, earlier you
saw ``\verb|Definition~\ref{def:COOKIE}|''; the ``\verb|~|'' ensures
that the reference number for COOKIE (\ref{def:COOKIE}) does not end
up at the beginning of a line.

To avoid having the last word in a paragraph on a line by itself, but
a \verb|~| between the last two words in the paragraph.  Caution: with
a \textbf{very} short paragraph, this may do more harm than good.  But
you should avoid very short paragraphs anyway (in most situations).

\need 1in

\section{Inter-Sentence Spacing}

In English typography, it is customary to have more space between the
word at the end of a sentence and the word at the beginning of the
next sentence.  Look at the first paragraph in this chapter and you can see
what I mean.   \TeX\ has a heuristic to automagically add this extra space
for you.

Namely, normally a ``\verb|.|'', ``\verb|?|'', or ``\verb|!|'' ends a
sentence when it is followed by white space (or it is at the end of a
line).  However, this rule is not always correct.  For example, in
``I saw H.M.~Schmoe downtown.\null'' the period after ``M'' is not a
sentence-ending period.  \TeX\ assumes this is the case because the
letter before the period is upper-case.

However, sometimes you might end a sentence after an abbreviation:
``Hamiltonian Cycle is in NP.''  To tell \TeX\ that this is a
sentence-ending period, you can separate the upper case letter from
the period as follows: ``\verb|NP\null.|''.

On the other hand, if you say ``My dog's name is Mr.~Smith'' you do
not want \TeX\ to treat ``Mr.\null'' as the end of a sentence.  In
this case if you write ``\verb|Mr.~Smith|'' then \TeX\ will use the
normal amount of inter-word space (and, as a bonus, you won't end up
with ``Mr.\null'' at the end of a line and ``Smith'' at the beginning
of the next line).

\section{Hyphens and Dashes}

There are four separate typographic symbols which you might be tempted
to use the same keyboard character for: ``-'', ``--'', ``---'' and ``$-$''.
The first is a hyphen, used for (wait for it\dots) hyphenated words,
such as ``multi-faceted''.  The second is known as an ``en-dash'', and
should be used when writing a numeric range such as ``40--45''.  The
third is known as an em-dash, and is used --- according to some styles
--- to set off parts of a sentence.  Finally, $-$ is an arithmetic
operator.  These four symbols are obtained with ``\verb|-|'',
``\verb|--|'', ``\verb|---|''and ``\verb|$-$|'', respectively.


\section{Clubs and Widows}


A \emph{widow} is a single line at the top of a page.  A \emph{club}
is a single line at the bottom of a page.  (Other people use different
names, including \emph{orphan}.  Moving on\dots)  I told you that so I
can tell you this.

Widows and clubs are typographically displeasing, and so by default
the \TeX\ typesetting engine has a moderate dislike for widows and
clubs and will try to break paragraphs (when spitting out a page) to
avoid them.  However, if you have lots of figures, tables, listings,
or other such non-textual stuff, or if you like to write short
paragraphs, you may find that \TeX/\LaTeX\ is not able to
automatically avoid them.

This is something you should only worry about when you are very close
to the final copy of your thesis, since minor edits can cause the
content of many following pages to move around.  When you are almost
done, if you want to deal with clubs and widows, read the blurb in
\verb|acadia-hon-thesis.sty| which gives you some information about
going this last typographic mile.
