\chapter{Formatting Points of Interest}
\label{chap:FORMATTING}

\section{One or Two Sides?}

You might have noticed that the left margin on odd pages is much
larger than the left margin on even pages.  This positions the text on
the page to allow the thesis to be printed on both sides of the paper
and bound.

If for some reason you wish the margins to be the same on both sides
of the page, look in \verb|header.tex| for the line
\verb|\documentclass[12pt,twoside,openright]{report}| and change
it to \verb|\documentclass[12pt]{report}|.

You may have also noticed that chapters (as well as the abstract, the
list of figures, and so on) always start on a right-hand (i.e.,
odd-numbered) page.  If you want a two-sided document but you are OK
with having a chapter start on a left-hand page (seriously??\null),
remove ``\verb|,openright|'' from the ``\verb|\documentclass|'' line.

If, when creating a two-sided document, you want the signature page to
be on the back of the title page (which is just so wrong), you will
need to break~(sic) things in the \verb|\firstThreePages| command in
the file \verb|acadia-hon-thesis.sty|.

%-----------------------------------------------------------------
% Section: Emphasis
%-----------------------------------------------------------------
\section{Emphasis}
In \LaTeX\ and \TeX\ spacing and text ``prettying'' (e.g.,
\textit{italics}, \textbf{bold}, etc.\null) are integral parts of
properly typesetting a document.  In a word processor, such issues are
often left up to the user, however, it is often advantageous to let
the typesetter do it for you.  For example, given the text (from
MacBeth):

\begin{quote}
They met me in the day of success: and I have\\
learned by the perfectest report, they have more in\\
them than mortal knowledge\dots
\end{quote}

\noindent how would you \emph{emphasize} a certain word?  In a word
processor you might choose to \textit{italicise} it using
\verb|\textit{success}|:

\begin{quote}
They met me in the day of \textit{success}: and I have\\
learned by the perfectest report, they have more in\\
them than mortal knowledge\dots
\end{quote}

\noindent But what if the quotation was instead written

\begin{quote}
\itshape
They met me in the day of \textit{success}: and I have\\
learned by the perfectest report, they have more in\\
them than mortal knowledge\dots
\end{quote}

\noindent Now \textit{italics} has no clear emphasis at all.  In
\LaTeX, we use the \verb|\emph{success}| command to indicate
emphasis.  If we emphasize \emph{success} in the previous quotations we
get: 

\begin{quote}
They met me in the day of \emph{success}: and I have\\
learned by the perfectest report, they have more in\\
them than mortal knowledge\dots
\end{quote}

\noindent but if the quotation was itself italicised

\begin{quote}
\itshape
They met me in the day of \emph{success}: and I have\\
learned by the perfectest report, they have more in\\
them than mortal knowledge\dots
\end{quote}

\noindent \verb|emph{...}| actually \textit{de-italicises} the
enclosed text, which ensures that it nonetheless stands out.  In a
word processor we would have to manually change the style of emphasis,
whereas \LaTeX\ takes care of it for us automagically.

%-----------------------------------------------------------------
% Section: Spacing
%-----------------------------------------------------------------
\section{Spacing}
Spacing is another contentious issue in \LaTeX. The compiler assumes
that most spacing is simply used to aid in reading the \LaTeX\ code,
rather than to provide whitespace in the typeset document.  For
example, the simple text:
\begin{verbatim*}
``We write text on 
    multiple lines with    random spacing      between
 words, and \LaTeX does the right 
                                     thing and groups it
                                     all together.''
\end{verbatim*}

\noindent results in 

\vspace{6 pt}

``We write text on 
    multiple lines with    random spacing      between
 words, and \LaTeX does the right 
                                     thing and groups it
                                     all together.''

\vspace{6 pt}

How about the space after ``\verb|\LaTeX|''\null? Well, this raises an
important point. Sometimes \LaTeX\ does \emph{not} seem to do the right
thing.  In this instance, it has ``eaten'' the space after the
command.  In cases like this we must enforce the space using
\verb*|\ |, where \verb*| | indicates a single space.  We can also use
this to force the compiler to leave a number of spaces.  Let's revisit
our example, with this in mind: 
\begin{verbatim*}
``We write text on 
    multiple lines with\ \ \ \ random spacing      between
 words, and \LaTeX\ does the right 
                                     thing and groups it
                                     all together.''
\end{verbatim*}

\noindent becomes:

\vspace{6 pt}

``We write text on 
    multiple lines with\ \ \ \ random spacing      between
 words, and \LaTeX\ does the right
                                     thing and groups it
                                     all together.''

\vspace{6 pt}

Observe that there is now a proper amount of space after ``\LaTeX''.
(Note: using constructs like ``\verb*|\ \ \ \ |'' is usually not the 
right way to leave some space in the middle of a line.  A much better way
is to use something like ``\verb|dog\hspace{1 in}cat|'', which
produces ``dog\hspace{1 in}cat'', keeping the dog exactly one inch
from the cat.)
                                     
\LaTeX\ and \TeX\ were not designed to be WYSIWYG (What You See Is
What You Get) systems, and with some experience using these systems
most people realize that is a Good Thing.  But what if we had actually
wanted to put a line-break in this text?  We can either use the
line-break delimiter \verb|\\|, or put two carriage returns in the
text, depending on exactly what we want.  Both \LaTeX\ and \TeX\ use
the convention that two consecutive carriage returns (i.e., a blank
line) denotes the start of a new paragraph.  Depending on the style of
your document (such as the Acadia thesis style), the first line of a
new paragraph may be indented, and there may be extra (vertical) space
between the two paragraphs.  (And, in some styles which normally
indent the first line of a paragraph, a paragraph immediately
following a section heading is not indented.  If you want to know why
this would be, talk to your local typographer.)

For example, in:
\begin{verbatim*}
``We write text on 
    multiple lines with\ \ \ \ \ random spacing      between
    
words, and \LaTeX\ does the right 
                                     thing and groups it
                                     all together.''
\end{verbatim*}

\noindent the blank line after ``between'' prompts the compiler
to indent the next line beginning with ``words\dots'':

\vspace{6 pt}

``We write text on 
    multiple lines with\ \ \ \ \ random spacing      between
    
words, and \LaTeX\ does the right 
                                     thing and groups it
                                     all together.''
\vspace{6 pt}

To indicate that you don't want to indent the first line of a new
paragraph, use the \verb|\noindent| command to start the new paragraph:
\begin{verbatim*}
``We write text on 
    multiple lines with\ \ \ \ \ random spacing      between
    
\noindent
 words, and \LaTeX\ does the right 
                                     thing and groups it
                                     all together.''
\end{verbatim*}

\noindent becomes:

\vspace{6 pt}

``We write text on 
    multiple lines with\ \ \ \ \ random spacing      between
    
\noindent
 words, and \LaTeX\ does the right 
                                     thing and groups it
                                     all together.''

\vspace{6 pt}

Some people get confused about when to use a blank line and when to
use \verb|\\|.  Simply, use a blank line if you are logically starting
a new paragraph, and use \verb|\\| if you want to start a new line
within the current paragraph.  Do NOT use \verb|\\| at the end of a
paragraph to get extra space.  There are many ways to get extra
vertical space; an example of one such command is
``\verb|\vspace{0.5 in}|'' which terminates the current paragraph (if you
give this command in the middle of a paragraph) and then puts out 0.5
inches of vertical space.

Having said that, there may be an unusual situation in which you want
to start a new line in the middle of a paragraph, and in such cases
you can use ``\verb|\\|''.  Also, when creating tables it is not
uncommon to use ``\verb|\\|'' (see Chapter~\ref{chap:FIGURESANDTABLES}).


                                     
\section{Controlling automatic spacing and hyphenation}

Sometimes you want to make sure that the spacing in your code is the
spacing that is used during the actual typesetting.  For example, suppose
we have a very long name that we don't want to be broken at the end of
a line (hyphenated):

``\dots\ the building blocks of life are often said to be the chemicals
Deoxyribonucleic acid (DNA) and Ribonucleic acid (RNA), but actually
amino acids are used to build DNA and RNA.'' 

Let's say we want to ensure that no matter what happens,
``Deoxyribonucleic acid (DNA)'' and ``Ribonucleic acid (RNA)'' are not
split over lines.  To do this we use the non-breakable space
``\verb|~|'' and a command to temporarily turn off hyphenation as follows:
\begin{verbatim}``\dots\ the building blocks of life are often said 
to be the chemicals {\hyphenchar\font=-1 Deoxyribonucleic~acid~(DNA)}
and  {\hyphenchar\font=-1 Ribonucleic~acid~(RNA)}, but actually 
amino acids are used to build DNA and RNA.''
\end{verbatim}
which produces

\vspace{6 pt}
``\dots\ the building blocks of life are often said to be 
the chemicals {\hyphenchar\font=-1 Deoxyribonucleic~acid~(DNA)} and 
{\hyphenchar\font=-1 Ribonucleic~acid~(RNA)}, but actually amino acids
are used to build DNA and RNA.''
 
\vspace{6 pt}
Of course in this example the results are rather ugly because
``\dots Deoxyribonucleic acid (DNA)\dots'' now hangs over the end of a
line, but you get the point.  You might want to re-word your sentence
(in the final version of your thesis) to avoid this problem.
